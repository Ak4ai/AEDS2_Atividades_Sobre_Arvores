\documentclass[conference]{IEEEtran}
\IEEEoverridecommandlockouts
\usepackage{cite}
\usepackage{amsmath,amssymb,amsfonts}
\usepackage{algorithmic}
\usepackage{graphicx}
\usepackage{textcomp}
\usepackage{xcolor}
\def\BibTeX{{\rm B\kern-.05em{\sc i\kern-.025em b}\kern-.08em
    T\kern-.1667em\lower.7ex\hbox{E}\kern-.125emX}}

\begin{document}

\title{Mapeamento e Correlação de Eventos Narrativos em Roteiros de Filmes: Análise de Metadados e Receptividade do Público\\
\thanks{Este trabalho foi financiado pela Agência de Pesquisa XYZ.}
}


\author{\IEEEauthorblockN{1\textsuperscript{st} Henrique de Freitas Araújo}
\IEEEauthorblockA{\textit{Engenharia da Computação} \\
\textit{CEFET-MG Campus V.}\\
Divinópolis, Brasil \\
hfreitas364outlook.com}
}

\maketitle

\begin{abstract}
Este artigo descreve o desenvolvimento de um sistema que mapeia eventos narrativos em roteiros de filmes, utilizando metadados para correlacionar esses eventos com a aceitação do público e notas de avaliação. O sistema analisa a estrutura narrativa dos filmes e relaciona-a com as avaliações recebidas, com o objetivo de explorar como a narrativa influencia a recepção crítica e popular. O artigo apresenta os métodos e resultados da implementação, além das implicações dessa análise para a compreensão da interação entre conteúdo cinematográfico e audiência.
\end{abstract}

\begin{IEEEkeywords}
Análise de Roteiros, Mapeamento de Eventos, Metadados, Receptividade do Público, Cinemática Narrativa.
\end{IEEEkeywords}

\section{Introdução}
A análise de roteiros de filmes tem sido amplamente utilizada em estudos de cinema e na indústria cinematográfica para entender o impacto da narrativa na recepção do público. Este trabalho propõe um sistema que mapeia eventos narrativos de diversos roteiros de filmes e usa metadados associados a esses filmes para correlacionar os eventos com a aceitação e notas atribuídas pelo público. O objetivo é investigar como diferentes tipos de eventos narrativos podem influenciar a percepção e a avaliação de um filme, com base em dados de resenhas, classificações e feedback de audiência.

\subsection{Geração da Base de Dados de Roteiros de Filmes e Metadados}

Para a criação da minha base de dados de roteiros de filmes, utilizei a ferramenta \textit{Movie Script Database}, desenvolvida por Aveek Saha, que coleta roteiros de filmes a partir de diversas fontes e gera metadados detalhados sobre os filmes, como título, data de lançamento e sinopse, extraídos dos bancos de dados TMDb e IMDb. A seguir, descrevo os passos principais adotados durante o processo de geração e integração dos dados.

Inicialmente, o programa permite a coleta dos roteiros de filmes de fontes como IMSDb, Dailyscript e Awesomefilm, entre outras. Para isso, configurei o arquivo \texttt{sources.json} para selecionar as fontes desejadas e executei o script \texttt{get\_scripts.py}, o que resultou em uma grande coleção de roteiros no formato \texttt{.txt}.

Em seguida, usei o script \texttt{get\_metadata.py} para coletar os metadados de cada filme, acessando as APIs do TMDb e IMDb, que fornecem informações como o título do filme, a data de lançamento e um resumo da trama. Esses dados foram posteriormente organizados e limpos através da remoção de duplicatas com o script \texttt{clean\_files.py}.

Como parte do processo de preparação para o estudo, realizei uma modificação manual nos metadados, adicionando as notas dos filmes. Essa etapa foi necessária pois as notas dos filmes eram um dado essencial para a análise de correlação entre o conteúdo dos roteiros e a recepção crítica, que é o foco deste estudo. Para isso, criei uma função personalizada chamada \texttt{vote\_average.py}, a qual foi responsável por inserir as notas diretamente no arquivo \texttt{clean\_parsed\_meta.json}, já organizado com os metadados dos filmes.

A função \texttt{vote\_average.py} utiliza informações de diversas fontes de notas, como IMDb e TMDb, para garantir que as avaliações sejam precisas e completas. Após a execução dessa função, as notas dos filmes foram corretamente associadas aos respectivos metadados e o arquivo final foi gerado, nomeado como \texttt{clean\_parsed\_vote\_meta.json}. Este novo arquivo contém os metadados dos filmes, agora enriquecidos com as notas, e está pronto para ser utilizado nas análises subsequentes.

O próximo passo foi a análise e a extração dos diálogos dos roteiros, realizada com o script \texttt{parse\_files.py}. Este script gerou arquivos com as falas dos personagens, organizados de forma a facilitar a análise linguística dos textos, com o objetivo de identificar padrões e características específicas dos roteiros, como a quantidade de diálogos por personagem.

Ao final, a estrutura do diretório foi organizada de forma a separar os arquivos de roteiros não processados, os arquivos temporários de formatos como PDF, os metadados limpos e os roteiros já processados, facilitando a navegação e o uso dos dados para análise subsequente.

Esses processos permitiram a criação de uma base de dados robusta e rica em metadados, que serve como o ponto de partida para a investigação proposta neste estudo.

\section{Objetivos}
O principal objetivo deste projeto é desenvolver uma aplicação computacional que mapeia e analisa eventos narrativos em roteiros de filmes, correlacionando-os com dados de aceitação do público. O sistema busca identificar padrões narrativos comuns em filmes com boas avaliações e explorar como os eventos narrativos podem afetar a experiência do espectador.

\section{Metodologia}
Para alcançar os objetivos propostos, foi desenvolvido um sistema que realiza as seguintes etapas:

\begin{itemize}
    \item \textbf{Mapeamento de Eventos}: Extração de eventos narrativos de roteiros de filmes, usando técnicas de processamento de linguagem natural (PLN).
    \item \textbf{Análise de Metadados}: Coleta de dados sobre cada filme, incluindo informações de avaliações e feedback do público.
    \item \textbf{Correlação de Eventos e Avaliações}: Uso de técnicas de análise de dados para correlacionar os eventos narrativos mapeados com as notas de aceitação do público.
\end{itemize}

A abordagem implementada utiliza uma base de dados com roteiros de filmes e informações de resenhas para identificar a relação entre a estrutura do roteiro e as avaliações dos filmes.

\section{Resultados}
Os resultados preliminares indicam que certos eventos narrativos, como viradas de trama e resolução de conflitos, têm um impacto significativo nas avaliações do público. Filmes com uma estrutura de roteiro bem definida, que respeitam os arcos narrativos típicos, tendem a ter melhores notas de avaliação. Além disso, eventos de suspense e tensão são correlacionados com um aumento nas classificações positivas.

\section{Conclusões}
Este estudo contribui para a compreensão da relação entre a estrutura narrativa de filmes e a receptividade do público. O sistema desenvolvido pode ser útil para cineastas, roteiristas e críticos cinematográficos ao fornecer uma ferramenta para analisar e prever a aceitação de filmes com base na análise de seus roteiros e eventos narrativos.

\section*{Agradecimentos}
Agradecemos à Universidade ABC pelo apoio técnico e financeiro, e aos desenvolvedores da biblioteca de PLN que possibilitaram a extração dos eventos narrativos dos roteiros.

\section*{Referências}
Cite as referências numericamente, como mostrado em \cite{b1}. Use o formato de citação IEEE, como mostrado nos exemplos abaixo.

\begin{thebibliography}{00}
\bibitem{b1} G. Eason, B. Noble, and I. N. Sneddon, ``On certain integrals of Lipschitz-Hankel type involving products of Bessel functions,'' Phil. Trans. Roy. Soc. London, vol. A247, pp. 529--551, April 1955.
\bibitem{b2} J. Clerk Maxwell, A Treatise on Electricity and Magnetism, 3rd ed., vol. 2. Oxford: Clarendon, 1892, pp. 68--73.
\end{thebibliography}

\end{document}
